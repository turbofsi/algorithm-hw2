\section{Problem I}


\begin{enumerate}[label=(\alph*)]
	\item 
		\textbf{Solution:} \\
		To get the expected number of recursive calls, we let $x_k$ be the indicator random variable associated with the event in which the $k$th element is chosen as $x^{*}$. Thus,
		$$ x_k = \begin{cases} 1 & $if the $k$th smallest element is chosen as $ x^{*}\\ 0 & $otherwise$ \end{cases} $$
		Let $T(n)$ denotes the expected number of recursive calls of a set of numbers with size of $n$. Thus, 
		$$ T(n) = \begin{cases} T(0) & $if the largest element is chosen as $ x^{*}\\ T(1) + T(0) & $if the second largest element is chosen as $ x^{*} \\ T(2) + T(0) & $if the third largest element is chosen as $ x^{*} \\ \cdot \\ \cdot \\ \cdot \\ T(n - 1) + T(0) & $if the smallest element is chosen as $ x^{*} \end{cases} $$
		Then we have,
		$$ T(n) = x_0T(0) + \sum_{k = 1}^{n - 1}x_k(T(k)+ T(0)) $$
		\begin{align*}
		E(T(n)) &= \frac{1}{n} \cdot T(0) + E(\sum_{k = 1}^{n - 1}x_k(T(k)+ T(0)))\\
		&= \frac{1}{n} \cdot T(0) + \sum_{k = 1}^{n - 1}\left\{[E(x_kT(k))] + E[x_kT(0)]\right\}\\
		&= \frac{1}{n} \cdot T(0) + \frac{1}{n}\sum_{k = 1}^{n - 1}E[T(k)] + \frac{n - 1}{n} \cdot T(0)\\
		&= T(0) + \frac{1}{n}\sum_{k = 1}^{n - 1}E[T(k)]
		\end{align*}

		Where $T(0) = 1$. So,
		$$ E(T(n)) = T(0) + \frac{1}{n}\sum_{k = 1}^{n - 1}E[T(k)] $$

	\item 
		\textbf{Solution:} \\
		According to the result we got from (a), we have:
		$$ \sum_{k = 1}^{n - 1}E[T(k)] = n \cdot (E(T(n)) - T(0)) $$
		Also, we have,
		$$ \sum_{k = 1}^{n - 2}E[T(k)] = (n - 1) \cdot (E(T(n - 1)) - T(0)) $$
		Substract two equations above we have:
		\begin{align*}
		E(T(n - 1)) &= n \cdot (E(T(n)) - T(0)) - (n - 1) \cdot (E(T(n - 1) - T(0))\\
		nE(T(n)) - nE(T(n - 1)) &= T(0) = 1\\
		E(T(n)) &= E(T(n - 1)) + \frac{1}{n}
		\end{align*}
		Thus, we have,
		\begin{align*}
		E(T(n)) &= E(T(n - 1)) + \frac{1}{n}\\
		E(T(n - 1)) &= E(T(n - 2)) + \frac{1}{n - 1}\\
		... &= ...\\
		E(T(2)) &= E(T(1)) + \frac{1}{2}
		\end{align*}
		Which could be simplified as, 
		$$ E(T(n)) = E(T(1)) + \sum_{i = 2}^{n} \frac{1}{i} $$
		Since $ E(T(1)) = 1 $,
		$$ E(T(n)) = \sum_{i = 1}^{n} \frac{1}{i} \leq lnn + 1 = O(logn)$$
	\item 
		\textbf{Solution:} \\
		Similar with (a), let $T(n)$ denotes the expected running time of a set of numbers with size of $n$. Thus, 
		$$ T(n) = \begin{cases} \theta(n) & $if the largest element is chosen as $ x^{*}\\ T(1) + \theta(n) & $if the second largest element is chosen as $ x^{*} \\ T(2) + \theta(n) & $if the third largest element is chosen as $ x^{*} \\ \cdot \\ \cdot \\ \cdot \\ T(n - 1) + \theta(n) & $if the smallest element is chosen as $ x^{*} \end{cases} $$
		Then we have,
		$$ T(n) = x_0\theta(n) + \sum_{k = 1}^{n - 1}x_k(T(k)+ \theta(n)) $$
		\begin{align*}
		E(T(n)) &= \frac{1}{n} \cdot \theta(n) + E(\sum_{k = 1}^{n - 1}x_k(T(k)+ \theta(n)))\\
		&= \frac{1}{n} \cdot \theta(n) + \sum_{k = 1}^{n - 1}\left\{[E(x_kT(k))] + E[x_k\theta(n)]\right\}\\
		&= \frac{1}{n} \cdot \theta(n) + \frac{1}{n}\sum_{k = 1}^{n - 1}E[T(k)] + \frac{n - 1}{n} \cdot \theta(n)\\
		&= \theta(n) + \frac{1}{n}\sum_{k = 1}^{n - 1}E[T(k)]
		\end{align*}
		So,
		$$ E(T(n)) = \theta(n) + \frac{1}{n}\sum_{k = 1}^{n - 1}E[T(k)] $$
	\item 
		\textbf{Solution:} \\
		Here, we can make a guess, $E(T(n)) \leq cn = O(n)$, where $c$ is a constant number. 
		By induction we have,
		\begin{align*}
		E(T(n)) &\leq \theta(n) + \frac{1}{n}\sum_{k = 1}^{n - 1}ck\\
		&\leq \theta(n) + \frac{c}{n}\sum_{k = 1}^{n - 1}k\\
		&\leq \theta(n) + \frac{cn}{2} = O(n)
		\end{align*}

	\item 
		\textbf{Solution:} \\
		 To compute the probability that Find-Max($X$) compares the $i$-th and the $j$-th largest items in $X$, we can think about when two items are not compared. For ease of analysis, we rename the elements of $X$ as $z_n, z_{n - 1}, ..., z_1$, with $z_i$ being the $i$th largest element. We can also define the set $Z_{ij} = \left\{ min(z_i, z_j), ..., z_2, z_1\right\}$ to be the set of elements between $z_1$ and minimum element of $z_i$ and $z_j$ , inclusive. because we assume that element values are distinct, once a $x^{*}$ is chosen with $z_j< x^{*} < z_i$ or $z_i < x^{*} < z_1$(we can suppose that $z_i > z_j$ for ease of analysis), we know that $z_i$ and $z_j$ cannot be compared at any subsequent time. If, on the other hand, $z_i$ is chosen as $x^{*}$ before any other item in $Z_{ij}$, then $z_i$ will be compared to each item in $Z_{ij}$ including $z_j$, except for itself. Similarly, if $z_j$ is chosen as $x^{*}$ before any other item in $Z_{ij}$, then $z_j$ will be compared to each item in $Z_{ij}$ including $z_i$, except for itself. Thus $z_i$ and $z_j$ are compared if and only if the first element to be chosen as $x^{*}$ from $Z_{ij}$ is either $z_i$ or $z_j$. Meanwhile, prior to the point at which an element from $Z_{ij}$ has been chosen as $x^{*}$, the whole $Z_{ij}$ is together in the same partition, so any element of $Z_{ij}$ is equally likely to be the first one chosen as $x^{*}$. Because the set $Z_{ij}$ has $j$ elements, and because $x^{*}$ is chosen randomly and independently, the probability that any given element is the first one chosen as $x^{*}$ is $1/j$. Thus, in general case, we have
		$$ Pr(z_i \text{ compared with } z_j) = Pr(z_i \text{ or } z_j \text{ is chosen as the first element from } Z_{ij}) = \frac{2}{max\{i, j\}}$$
\end{enumerate}