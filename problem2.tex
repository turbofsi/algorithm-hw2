\section{Problem II}

\textbf{Solution:} \\
To show that number of comparisons is at least $n - k + log\dbinom{n}{k - 1}$ is equivalent to showing that number of outcome boxes is at least $2^{n - k} \times \dbinom{n}{k - 1}$.\\\\

Let $V(n, k)$ denote the number of comparisons of finding the $k$th largest element of $n$-element set. \\\\

When $k = 1$, $V(n, 1)$ represent the number of comparisons for finding the largest element of $n$-element set. We could easily observe that $V(n, 1) \geq n - 1$, since every element except the largest must lose at lease one comparison. This observation implies that in any comparison tree to find the largest element, every leaf has depth at least $n - 1$, which implies that there must be at least $2^{n - 1}$ leaves. We can generalize this argument to prove a lower bound for $V(n, k)$ for arbitrary values of $k$.\\\\

Let $T$ be a comparison tree that identifies the $k$th largest element $x_{(k)} \in X$.\\\\

Suppose we are at some outcome box for determing the $k$th largest element $x_{(k)}$. Assume there are some elements not yet directly known to be bigger or smaller than $x_{(k)}$. \\

\begin{itemize}
	\item Set $U$ is set of elements not yet directly known to be bigger or smaller than $x_{(k)}$.
	\item Set $L$ is set of those known to be larger than $x_{(k)}$
	\item Set $S$ is set of those known to be smaller than $x_{(k)}$
\end{itemize}
Suppose that there is an element $x_* \in U$ such that $x_* > x_{(k)}$, which implies that $\mid L \mid = k - 1 + 1 = k$ and $x_{(k)}$ become the $k + 1$ largest element. However, we have known that $x_{(k)}$ is the $k$th largest element which is a contradiction. So we can conclude that when a decision reaching the right outcome of finding the $k$th largest element also decide correctly which the $k - 1$ larger elements are.\\\\
Now suppose that the set of $k - 1$ largest elements of $X$:
$$ L = \left\{x_{(1)}, x_{(2)}, ... ,x_{(k-1)}\right\} $$  
Since those elements in set $L$ must bigger than those elements in set $S$, we could remove those comparisons from $T$. Call the reduced tree $R$. Since the reduced tree $R$ also identifies the largest element of $S$, based on the conclusion we got from base case $k = 1$, $R$ must have at least $2^{n - k}$ leaves. Since there are $\dbinom{n}{k - 1}$ choices for set $L$, we conclude that $T$ has at least $\dbinom{n}{k - 1} \cdot 2^{n - k}$ leaves which also implies that the number of comparisons is at least $n - k + log\dbinom{n}{k - 1}$. 







